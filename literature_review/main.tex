\documentclass[12pt]{article}

\usepackage{lmodern}
\usepackage[margin=1in]{geometry}

\title{\vspace{-1in}\huge{Literature Review}}
\author{Ahmed Shuaib}
\date{\today}

\begin{document}
\maketitle
\vspace{-0.5in}
\section*{}
There are many ways of incorporating automata in game design that have been considered. Many design decisions take place during the development of a game, these decisions can vary based on the genre of the game.
\par
When developing 2D games that fall in the genre of role-playing games (RPGs), adventure games, and puzzles. A procedural-level generation has been demonstrated using cellular automata. In a thesis from 2022, A genetic algorithm is used to evolve the cellular automata rules applied to generate game levels. It is important to note that a procedural content generator designed using this method is meant to be used during the game development process rather than at runtime. The resulting game levels can have a significant impact on a game's replayability, engagement, and scalability. [1]
\par
Games designed solely using automata are few in number, but they do exist. Most games designed using automata are finite in nature, however, in a paper from 2016 students designed an infinite runner game using mealy machines. The game consisted of several states: running, jumping, flying, and game over. The paper concluded that games designed using automata are less prone to bugs and the development process is simplified by handling many implementation decisions in the design phase. [2]
\par
An example of a game designed solely using automata is Conway's Game of Life. Developed by John Conway the Game of Life is a "zero-player" game utilizing cellular automata. The game is essentially a square grid containing cells that evolve based on three rules: birth, death, and survival. A dead cell will become alive in the next "time step" if three of its eight neighboring cells are alive. A cell can die in the next time step due to overcrowding, which occurs when it has four or more alive neighbors, or from exposure, which happens when it has one or fewer alive neighbors. A cell survives the current "time step" if the cell has either two or three alive neighbors. [3]
\par
Incorporating automata into game design offers a versatile approach. Allowing design decisions tailored to each game's genre. Merging the best of not only the gameplay experience but also the design and development.

\section*{\small References}
\begin{enumerate}
    \item [{[1]}] Adel Sabanovic and Amir Khodabakhshi, Evolved cellular automata for 2D video game level generation, 2022.
    \item [{[2]}] Abid Jamil, Engr. AsadUllah and Mohsin Rehman, An Infinite Runner Game Design using Automata Theory, 2016.
    \item [{[3]}] Kuldeep Vayadande, Ritesh Pokarne, Tanmay Patil, Mahalakshmi Phaldesai, Prachi Kumar, and Tanushri Bhuruk, Simulation of Conway's Game of Life using cellular automata, 2022.
\end{enumerate}
\end{document}

